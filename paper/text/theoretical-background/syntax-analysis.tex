\section{Syntax Analysis}

Syntax analysis, commonly known as parsing, is the second phase of the compilation process. It operates on the token stream produced by the lexical analyser and fulfils the following objectives \cite{aho2007compilers}:

\begin{itemize}
\item \textbf{Syntax Verification:} Ensure the token sequence conforms to the language grammar.
\item \textbf{Structural Construction:} Build a hierarchical representation (parse tree or Abstract Syntax Tree).
\item \textbf{Error Reporting:} Detect and report syntactic errors in the source code.
\end{itemize}

The parser examines the relationships and order of tokens to confirm that the source code adheres to the defined syntax. A successful parse yields a structured representation, which is essential for subsequent phases such as semantic analysis and intermediate code generation.

\subsection{Context-Free Grammars (CFGs): Defining Syntax}

Context-free grammars (CFGs) formally specify the syntax of programming languages \cite{aho2007compilers}. A CFG consists of:

\begin{itemize}
\item \textbf{Terminal Symbols:} Basic tokens provided by the lexical analyser.
\item \textbf{Non-terminal Symbols:} Higher-level constructs (e.g. expressions, statements).
\item \textbf{Production Rules:} Definitions of how non-terminals expand into terminals and/or other non-terminals.
\item \textbf{Start Symbol:} The root non-terminal representing a complete program.
\end{itemize}

Derivation begins at the start symbol and proceeds by applying production rules until the input token sequence is generated. Table~\ref{tab:parse-tree-general} shows the explanatory text and a general parse tree side by side.

\begin{table}[ht!]
    \centering
    \renewcommand{\arraystretch}{1.2} 
    \begin{tabularx}{\textwidth}{%
        >{\raggedright\arraybackslash}X%
        >{\centering\arraybackslash}X%
      }
      \textbf{Explanation} 
        & \textbf{General Parse Tree} \\
      \hline
      A CFG derivation can be visualised as a tree: the start symbol is at the root, and each application of a production rule adds children corresponding to the symbols on the right-hand side. For instance, given:
      \[
        S \;\rightarrow\; A\,B,\quad A\;\rightarrow\;a,\quad B\;\rightarrow\;b,
      \]
      the parse tree captures exactly that structure, showing how \(S\) expands into \(A\) and \(B\), and then into terminals \(a\) and \(b\). 
        & 
      \begin{tikzpicture}[level distance=1.2cm, sibling distance=1.2cm]
        \node {S}
          child { node {A}
            child { node {a} }
          }
          child { node {B}
            child { node {b} }
          };
      \end{tikzpicture} \\
    \end{tabularx}
    \caption{General parse tree for the grammar \(S \rightarrow AB,\;A\rightarrow a,\;B\rightarrow b\).}
    \label{tab:parse-tree-general}
\end{table}  

\subsection{Top-Down Parsing Techniques}

Top-down parsing begins at the start symbol and works towards the input tokens, predicting productions to derive the input. Common methods include:

\subsubsection*{Recursive Descent Parsing}
Each non-terminal is implemented as a recursive function. Functions attempt to match tokens and call other functions for nested constructs. Backtracking may be required if a prediction fails \cite{aho2007compilers}.

\subsubsection*{LL(1) Parsing}
LL(1) parsers scan left-to-right, construct a leftmost derivation, and use one lookahead symbol. They rely on FIRST and FOLLOW sets and parsing tables to avoid backtracking \cite{aho2007compilers}.

\subsection{Bottom-Up Parsing Techniques}

Bottom-up parsing starts with tokens and reduces them to non-terminals, reconstructing a rightmost derivation in reverse. Key approach:

\subsubsection*{LR (Shift-Reduce) Parsing}
LR parsers use ACTION and GOTO tables to decide between \textit{shift} and \textit{reduce} operations. Variants include SLR, LALR, and LR(1) \cite{aho2007compilers}.


\begin{table}[h!]
    \centering
    \renewcommand{\arraystretch}{1.2} % increase row height
    \begin{tabularx}{\textwidth}{|
        >{\raggedright\arraybackslash}p{0.2\textwidth}|
        >{\raggedright\arraybackslash}X|
        >{\raggedright\arraybackslash}X|
      }
      \hline
      \textbf{Feature} & \textbf{Top-Down Parsing} & \textbf{Bottom-Up Parsing} \\
      \hline
      Starting Point
        & Start symbol (root of parse tree)
        & Input tokens (leaves of parse tree) \\
      \hline
      Derivation Type
        & Leftmost derivation
        & Rightmost derivation in reverse \\
      \hline
      Handling of Left Recursion
        & Requires elimination of left recursion
        & Can handle left-recursive grammars \\
      \hline
      Common Techniques
        & Recursive Descent, LL(1) Parsing, Predictive Parsing
        & LR Parsing (SLR, LALR, LR), Shift-Reduce Parsing, Operator Precedence Parsing \\
      \hline
      Prediction of Production
        & Predicts the next production to apply
        & Postpones the decision until the entire right side is seen \\
      \hline
      Control
        & Explicitly constructs the parse tree from the top
        & Attempts to reduce the input to the start symbol \\
      \hline
    \end{tabularx}
    \caption{Comparison of Top-Down and Bottom-Up Parsing}
    \label{tab:td-vs-bu}
  \end{table}


\pagebreak
